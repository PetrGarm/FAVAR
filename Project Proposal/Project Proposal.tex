\documentclass[a4paper, 14pt]{article}
\usepackage[utf8]{inputenc}
\usepackage{amsmath,amsfonts,amssymb,amsthm,mathtools} % AMS
\usepackage{wrapfig,lipsum, cleveref}
\usepackage{icomma} 
\usepackage{geometry} 
\usepackage{longtable}
\usepackage{booktabs}

\geometry{top=25mm}
\geometry{bottom=25mm}
\geometry{left=25mm}
\geometry{right=25mm}

\linespread{1.5}

\setlength{\parindent}{5ex}


%% Шрифты
\usepackage{euscript}	 % Шрифт Евклид
\usepackage{mathrsfs} % Красивый матшрифт


%% Цитирование и библиография
\usepackage{natbib}

%% Свои команды
%\DeclareMathOperator{\sgn}{\mathop{sgn}}

%% Перенос знаков в формулах (по Львовскому)
\newcommand*{\hm}[1]{#1\nobreak\discretionary{}
{\hbox{$\mathsurround=0pt #1$}}{}}


\title{Factor Augmented VAR for Forecasting Russian Macroeconomic Time Series}
\usepackage{cmap}					% поиск в PDF
\usepackage[T2A]{fontenc}			% кодировка
\usepackage[utf8]{inputenc}			% кодировка исходного текста
\usepackage[english]{babel}			% локализация и переносы
\usepackage{graphicx}
\graphicspath{{pictures/}}
\DeclareGraphicsExtensions{.pdf,.png,.jpg}
\author{Petr Garmider}
\date{\today}
\begin{document}
	\newpage
	\thispagestyle{empty}
	\begin{center}
		
		\vspace{0.1ex}
		
		{\textbf{NATIONAL RESEARCH UNIVERSITY HIGHER SCHOOL OF ECONOMICS}}\\
		\vspace{1ex}
		{\textbf{The Faculty of Economics Scientce}}\\
		\vspace{1ex}
		{\textbf{
				Department of Applied Economics}}\\
		
	\end{center}
	\vspace{5ex}
	\begin{center}
		\vspace{3ex}
		{\textbf{PROJECT PROPOSAL}}\\
		\vspace{3ex}
		{
			\vspace{2ex} \textbf{FACTOR-AUGMENTED VECTOR AUTOREGRESSION FOR FORECASTING RUSSIAN MACROECONOMIC TIME SERIES.}}
	\end{center}
	\begin{flushright}
		\vspace{13ex}
		\noindent
		\textit{Petr Garmider, BEC 165}
		\\
		\vspace{5ex}
		Advisor:\\
		\vspace{2ex}
		
		\textit{Boris Demeshev, \\Senior Lecturer, \\Department of Applied Economics}\\
		
		
	\end{flushright}

	\vspace{64ex}
	
	\begin{center}
		\vspace{3ex}
		{Moscow}\\
		\vspace{1ex}{28 February 2020}
	\end{center}	
	
	\newpage
	
\section*{Introduction}

Forecasting of macroeconomic time series is an important task for different economic institutions. Accurate forecasts for main macro indicators allows central banks timely react to possible dangers for the economy. Proper actions conducted by central bank may prevent a possible recession or the same way, wrong actions may result in initiating one. Due to the fact that some of its instruments have a delay time after its decision will have an impact on the economy, central banks especially interested in forecasts to be able influence a situation while it is possible. 

Macro indicators forecasts are also useful for commercial companies as well. There are many scenarios in which corporations are particularly interested in understanding of possible future outcomes for macro indicators, such as: choosing the right price policy, signing long-term contracts with counterparties, making decision on a possibility of entering a new market and etc. One cannot deny that a huge amount of financial instruments heavily dependent on macro factors. Investors in attempts to value a particular financial asset always build a forecast of its determinants. By this manner, almost everything to some extent shows comovement with one or another macro factor. This demonstrates that forecasting of macroeconomic time series is extremely crucial part in quite huge range of spheres.  

There are a number of models used for time series prediction. Each model uses different approach of dealing with time series. Some approaches do not require any assumptions about data and are aimed to minimise proper loss function in specific manner. The other includes restrictions on data, assuming, for example, presence of data-generating process with parameters that are to be estimated. State-of-the-art models, usually employs mixed approach such as forecast is based on results of two mentioned methods. Models of the first type show high accuracy of prediction, however fail to interpret its results and prone to outliers. Models of the second type, on the contrary, show moderate accuracy, however give an information about moving forces and produce quite robusts results. The last approach, as one can understand tries to balance between advantages and disadvantages of such approaches. Of course, there are plenty of other models with different view on a data. 

One can divide second type of models onto two subcategories: univariate and multivariate approaches. To make a forecast for considering time series more accurate, the last method tries to use additional data, while first assumes data generating process of a particular series to be a function of past realisations of itself and only. This paper mostly deals with multivariate models: vector auto regression (VAR) suggested by \cite{sims1980martingale} and its variation factor-augmented VAR that is aimed to solve drawbacks of classical version.  

In this paper I will consider factor-augmented vector autoregression (FAVAR): method of estimation, its application for forecasting purposes and accuracy evaluation of produced forecast and its robustness to outliers. There is no academic literature that uses FAVAR model for Russian macroeconomic time series prediction. The goal is to understand, whether factor model outperform univariate approach. One also may be interested in comparing classical VAR with its augmented version. It will be shown, that VAR is restricted model of FAVAR, therefore it is possible to test whether difference between the two is statistically significant. 
 
\section*{Literature review}
There are several studies that discuss FAVAR model and employs its potential for particular purposes. \cite{bernanke2005measuring} were first to introduce a model to incorporate a huge set of information to assess structural shocks effect on monetary policy. They discuss \cite{sims1992interpreting} and his interpretation of "price puzzle", when several works showed that VAR models predict that contractionary monetary policy shock is followed by rise in price level, rather than decrease as traditional macroeconomic theory suggests. Authors believe that explanation proposed by the inventor of VAR still has weak sides, therefore suggest their own way to deal with "price puzzle". In the paper authors used 120 monthly time series pointing out that central banks while making decisions on interest rate act in "data rich environment", considering a larger set of time series than standard VAR approach allows to include. Therefore, econometrician, ignoring this fact, faces with a problem of biased estimates. They suggested two ways of how FAVAR can be estimated and one of them will be presented as pseudo-code in this work. The main goal of model inventors was to obtain correct impulse responses for variables of interest rather than producing forecasts for them. 

Nevertheless, there are some works focused on applying considering model for prediction task as well, for example,  \cite{monch2008forecasting} in his work uses FAVAR to forecast yield curve. Authors mentions that state-of-the-art models in this sphere have a large number of parameters to estimate, which means that user cannot simply include as many explanatory variable as one wishes, so there is a risk to lose important information. Therefore, author used approach of latent factor extraction described in \cite{bernanke2005measuring} and mixed this with traditional models for yield curve prediction. 

What for the time series prediction, most of the papers agree, that factor-augmented VAR shows better performance for long-term forecasts. In \cite{figueiredo2013forecasting} FAVAR is used for time series forecasting of Brazilian consumer inflation (IPCA). Authors found that factor based models outperform those of traditional AR ones on long horizons. They also considered estimation methods pointing out that ML approach strictly outperforms two-step PCA method, that is the focus of study in this paper. \cite{pang2010forecasting} employed factor model for predicting GDP growth rate, unemployment and inflation rates of the Hong Kong economy. Author's results are in line with those obtained in \cite{figueiredo2013forecasting} -- FAVAR performs better than simple VAR and AR models especially for long horizons. Author used 76 mothly variables that are believed to reflect economic situation in Hong Kong. Concerning to data transformation, author followed methodology suggested in \cite{stock2005implications}.


\newpage
\bibliography{bibliography}
\bibliographystyle{APA}
\end{document}	